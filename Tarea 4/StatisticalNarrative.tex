\documentclass{article}\usepackage[]{graphicx}\usepackage[]{color}
%% maxwidth is the original width if it is less than linewidth
%% otherwise use linewidth (to make sure the graphics do not exceed the margin)
\makeatletter
\def\maxwidth{ %
  \ifdim\Gin@nat@width>\linewidth
    \linewidth
  \else
    \Gin@nat@width
  \fi
}
\makeatother

\definecolor{fgcolor}{rgb}{0.345, 0.345, 0.345}
\newcommand{\hlnum}[1]{\textcolor[rgb]{0.686,0.059,0.569}{#1}}%
\newcommand{\hlstr}[1]{\textcolor[rgb]{0.192,0.494,0.8}{#1}}%
\newcommand{\hlcom}[1]{\textcolor[rgb]{0.678,0.584,0.686}{\textit{#1}}}%
\newcommand{\hlopt}[1]{\textcolor[rgb]{0,0,0}{#1}}%
\newcommand{\hlstd}[1]{\textcolor[rgb]{0.345,0.345,0.345}{#1}}%
\newcommand{\hlkwa}[1]{\textcolor[rgb]{0.161,0.373,0.58}{\textbf{#1}}}%
\newcommand{\hlkwb}[1]{\textcolor[rgb]{0.69,0.353,0.396}{#1}}%
\newcommand{\hlkwc}[1]{\textcolor[rgb]{0.333,0.667,0.333}{#1}}%
\newcommand{\hlkwd}[1]{\textcolor[rgb]{0.737,0.353,0.396}{\textbf{#1}}}%

\usepackage{framed}
\makeatletter
\newenvironment{kframe}{%
 \def\at@end@of@kframe{}%
 \ifinner\ifhmode%
  \def\at@end@of@kframe{\end{minipage}}%
  \begin{minipage}{\columnwidth}%
 \fi\fi%
 \def\FrameCommand##1{\hskip\@totalleftmargin \hskip-\fboxsep
 \colorbox{shadecolor}{##1}\hskip-\fboxsep
     % There is no \\@totalrightmargin, so:
     \hskip-\linewidth \hskip-\@totalleftmargin \hskip\columnwidth}%
 \MakeFramed {\advance\hsize-\width
   \@totalleftmargin\z@ \linewidth\hsize
   \@setminipage}}%
 {\par\unskip\endMakeFramed%
 \at@end@of@kframe}
\makeatother

\definecolor{shadecolor}{rgb}{.97, .97, .97}
\definecolor{messagecolor}{rgb}{0, 0, 0}
\definecolor{warningcolor}{rgb}{1, 0, 1}
\definecolor{errorcolor}{rgb}{1, 0, 0}
\newenvironment{knitrout}{}{} % an empty environment to be redefined in TeX

\usepackage{alltt}
\IfFileExists{upquote.sty}{\usepackage{upquote}}{}
\begin{document}
\raggedright

{ \setlength{\parskip}{3mm} \Large Statistical Narrative Telling Compelling Stories with Numbers.

Nombre: Juan Carlos D\'iaz Flores}

{\setlength{\parskip}{7mm} \large Es importante destacar la diferencia entre una gr\'afica de dispersi\'on y una de burbujas; la primera, relaciona las variables ayudando a observar que relaci\'on existe entre ellas, la segunda, es muy parecida a la primera solo que agrega un tama\~no a los puntos, es decir, agrega una variable extra que es la dimensi\'on de lo que estamos midiendo (variable de tama\~no), y m\'as todav\'ia ya que este tipo de gr\'aficas tienen una caracter\'istica muy peculiar, es el movimiento, y este representa el cambio a trav\'es del tiempo.}

{\setlength{\parskip}{7mm} \large Los relatos se pueden expresar de diferentes maneras por ejemplo: de forma visual, por medio del habla, etc., gracias a que la tecnolog\'ia se encuentra en expansi\'on, es posible captar el significado gracias a im\'agenes, sonidos, etc., y cada una de ellas debe utilizarse de la mejor forma, es decir, relacionarse con el relato para hacer que pueda ser captado en su totalidad. }

{\setlength{\parskip}{7mm} \large El autor habla acerca de varias t\'ecnicas para poder entender el relato, es decir, que comprenden la estad\'istica narrativa, por ejemplo: como primer punto simplicidad, es necesario conocer el n\'ucleo, resumir lo principal y tratar de explicar de forma sencilla, y tambi\'en tener cuidado en no agregar informaci\'on extra que no corresponda con el sentido del relato, como segundo punto sin fisuras, no solo debemos utilizar palabras, es necesario combinar estas con im\'agenes y gr\'aficas en el documento para de esta forma lograr un contenido no tan aburrido y m\'as din\'amico, es importante que si por ejemplo, queremos utilizar una gr\'afica para mostrar datos, no es necesario mezclar texto con la imagen y repetir lo que dice el texto, es m\'as efectivo solo poner la gr\'afica y explicar con palabras para tener la concentraci\'on del p\'ublico, como tercer punto informativo, es una buena opci\'on agregar informaci\'on que resulte nueva para los oyentes, de esta forma no hacer aburrido el relato, como cuarto punto verdadero, hay que utilizar fuentes cre\'ibles de informaci\'on, tratar de sonar cre\'ibles durante la exposici\'on para que la audiencia crea la informaci\'on como verdadera, quinto punto contextual, los n\'umeros juegan un papel muy importante pero deben ser utilizados con cuidado, es decir, no aislarlos y presentarlos de forma aislada, hay que mezclarlos con informaci\'on para tenerlos en contexto, no olvidando relacionar cifras con texto para lograr que obtengan un mayor significado, sexto punto familiar, hay que acoplarse a el tipo de p\'ublico y tratar de explicarlo en t\'erminos conocidos por ellos para que les resulte familiar, tener precauci\'on al momento de querer impactar a la audiencia con alguna informaci\'on siempre teniendo en mente los puntos anteriores, s\'eptimo punto concreto, es importante ser preciso y no tan abstracto ya que la capacidad de abstracci\'on en cada persona es diferente, podemos esconder cierta informaci\'on y esconderla mediante abstracciones a fin de compartir m\'as datos teniendo en cuenta que hay que hacerlo sin caer en los extremos, es decir, obvio o indescifrable, octavo punto personal, independientemente de la historia tenemos que saber de que forma \'esta, aborda sus intereses, como noveno ounto emocional, los sentimientos juegan un papel muy importante, s\'i logramos que las personas tengan alguna emoci\'on y s\'i relacionamos el discurso con lo que ellas viven d\'ia a d\'ia, es muy probable que tengamos su atenci\'on por el resto del discurso, como d\'ecimo punto procesable, para lograr un efecto en nuestra historia debemos relacionar la forma en que ense\~namos y la manera en como ponemos a trabajar la ense\~nanza, onceavo punto secuencial, los relatos normalmente tienen un orden cronol\'ogico, se presentan en su momento adecuado para tener una mejor significado, y si se utilizan las diapositivas correctamente se pueden complementar las palabras del orador con im\'agenes y darle cierta veracidad a lo platicado, aunque hay que utilizarlas con mucho cuidado; para ense\~nar una historia los elementos a mostrar ya sea en texto o gr\'afica deben ense\~narse secuencialmente, de esta forma el oyente entender\'a que adem\'as de los datos se desea mostrar algo m\'as, que es la historia.}

{\setlength{\parskip}{7mm} \large Como conclusi\'on me parece que la estad\'istica narrativa es una forma muy eficaz de presentar datos y relacionarlos con otras variables, es una mezcla de las distintas formas en las que se pueden compartir datos o informaci\'on con alg\'un p\'ublico.}


\end{document}
